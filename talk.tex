\documentclass{beamer}

\include{mathdefs}
\newcommand{\cp}{$\text{CP}^{N-1}$\,\,}

\usefonttheme{serif}
\usetheme{AnnArbor}

\setbeamertemplate{background canvas}[vertical shading][bottom=yellow!80,top=yellow!30]
\setbeamertemplate{blocks}[rounded][shadow=true]

\setbeamercolor{normal text}{fg=blue!80}
\setbeamercolor{alerted text}{fg=orange!90}
\setbeamercolor{math text}{fg=red!90}

\title[Effective Action of \cp]
      {Low-Energy Effective Action\\
       of \cp Model at large $ N $}

\author{Pavel A. Bolokhov}

\date{April 3, 2015}

\institute[SFU \& SPbSU]{FTPI ~~$\cdot$~~ University of Minnesota}

\begin{document}

\maketitle

%%%%%%%%%%%%%%%%%%%%%%%%%%%%%%%%%%%%%%%%%%%%%%%%%%%%%%%%%%%%%%%%%%
%%%%%%%%%%%%%%%%%%%%%%%%%%%%%%%%%%%%%%%%%%%%%%%%%%%%%%%%%%%%%%%%%%
\section{Introduction}
%%%%%%%%%%%%%%%%%%%%%%%%%%%%%%%%%%%%%%%%%%%%%%%%%%%%%%%%%%%%%%%%%%
%%%%%%%%%%%%%%%%%%%%%%%%%%%%%%%%%%%%%%%%%%%%%%%%%%%%%%%%%%%%%%%%%%
\begin{frame}{}
\usefont{T1}{ptm}{m}{n}\fontsize{60pt}{60pt}\selectfont
\begin{center}
        Introduction
\end{center}
\end{frame}


%%%%%%%%%%%%%%%%%%%%%%%%%%%%%%%%%%%%%%%%%%%%%%%%%%%%%%%%%%%%%%%%%%
\begin{frame}{Bosonic theory}

	We address the task of solving the two-dimensional \cp theory

	The non-supersymmetric \cp describes a complex vector 
\[
	n^l\,,\qquad\qquad l ~=~ 1\,,\, ...\,,\, N
\]
	subject to the identification
\[
	\vec{n} ~~\sim~~ \lambda\, \vec{n}\,,
	\qquad\qquad \lambda ~\in~ \mathcal{C}
\]

	In the physical setting $ \vec{n} $ arises as {\it orientational} moduli 
	living on the worldsheet of non-Abelian vortices (with a gauge group SU($ N $))

\end{frame}


%%%%%%%%%%%%%%%%%%%%%%%%%%%%%%%%%%%%%%%%%%%%%%%%%%%%%%%%%%%%%%%%%%
\begin{frame}{}

	The {\it gauge} formulation for such a theory was introduced by Witten
\[
 	\mc{L} ~~=~~
		\,\frac{1}{4e^2}\,F_{kl}^2  ~+~  \frac{1}{2e^2}\, D^2  ~+~
		\big| \nabla n \big|^2  ~+~  iD \big(\, \big|n^l \big|^2 - 2\beta \,\big)
\]

	where 
\[
	\nabla_k\, n^l  ~~=~~  \left(\, \p_k ~-~ i\,A_k \,\right)\, n^l
\]
	In the limit $ e \,\to\, \infty $ resolution of $ A_k $ and $ D $ imposes
	the CP$^{N-1}$ constraint $ \vec{n} ~\sim~ \lambda\, \vec{n} $

	One of $ n^l $ components can be expressed in terms of the other $ N - 1 $,
	and put to an arbitrary phase --- {\it e.g.} set {\it real}
\[
	\mc{L}  ~~=~~  \big|\, \p\, n \,\big|^2  ~+~  \big(\, \ov{n}\, \p_k \, n \,\big)^2\,,
	\qquad\qquad l ~=~ 1\,,\, ...\,,\, N-1
\]

\end{frame}


%%%%%%%%%%%%%%%%%%%%%%%%%%%%%%%%%%%%%%%%%%%%%%%%%%%%%%%%%%%%%%%%%%
\begin{frame}{\ntwot Supersymmetric Theory}

\begin{align}
% 
\notag
 	\mc{L}_\text{(2,2)} & ~~=~~
	\,\frac{1}{4e^2}\,F_{kl}^2  ~+~ \frac{1}{e^2} \big|\p_k \sigma\big|^2 
	~+~ \frac{1}{2e^2}\, D^2 ~+~
	\\[2mm]
%
\notag
	&
	~~+~~
	\frac{1}{e^2}\, \blar\, i\p_L\, \lar  ~+~  \frac{1}{e^2} \blal\, i\p_R\, \lal
	~+~
	\\[2mm]
%
\notag
	&
	~~+~~
	\big| \nabla n \big|^2  ~+~ | \sqrt{2}\sigma |^2 \big| n^l \big|^2
	~+~ iD \big( \big|n^l \big|^2 - 2\beta \big)
	~+~
	\\[2.8mm]
%
\notag	&
	~~+~~ \bxir\, i\nabla_L \xir  ~+~ \bxil\, i\nabla_R \xil ~+~
	i\sqrt{2}\sigma\, \bxir \xil
	~+~ i\sqrt{2}\ov{\sigma}\, \bxil \xir
	~+~
	\\[2.8mm]
%
\notag
	&
	~~+~~ i\sqrt{2}\, \ov{\xi_{[R}\, \lambda}{}_{L]}\, n
	~-~ i\sqrt{2}\, \nbar\,  \lambda_{[R}\, \xi_{L]}\,,
	\qquad
	l  ~=~  1,\,...\,N
\end{align}

\end{frame}


%%%%%%%%%%%%%%%%%%%%%%%%%%%%%%%%%%%%%%%%%%%%%%%%%%%%%%%%%%%%%%%%%%
\begin{frame}{The Exact Superpotential}
	
	This theory is known to have an exact {\it Veneziano-Yankielowicz} type
	superpotential
\[
	\int\, 	d\theta_R\, d\bar\theta_L \left( \sqrt{2}\Sigma\, \log \sqrt{2}\Sigma ~-~  
					         \sqrt{2}\Sigma \right)
\]
	also known as {\it Witten} superpotential,
	where $ \Sigma $ is a {\it twisted} superfield
\[
	\Sigma     ~~=~~    \sigma  ~~-~~  \sqrt{2}\, \theta_R \ov\lambda{}_L
				    ~~+~~  \sqrt{2}\, \ov\theta{}_L \lambda_R
				    ~~+~~  \sqrt{2}\, \theta_R \ov\theta{}_L \lgr D ~-~ i\, F_{03} \rgr
\]	
	and
\[
	\Sigma    ~~=~~    \frac{i}{\sqrt 2}\, D_L\, \ov D{}_R\, V
\]

	We show that at large $ N $ one can do better than just superpotential

\end{frame}




%%%%%%%%%%%%%%%%%%%%%%%%%%%%%%%%%%%%%%%%%%%%%%%%%%%%%%%%%%%%%%%%%%
%%%%%%%%%%%%%%%%%%%%%%%%%%%%%%%%%%%%%%%%%%%%%%%%%%%%%%%%%%%%%%%%%%
\section{The Effective Potential}
%%%%%%%%%%%%%%%%%%%%%%%%%%%%%%%%%%%%%%%%%%%%%%%%%%%%%%%%%%%%%%%%%%
%%%%%%%%%%%%%%%%%%%%%%%%%%%%%%%%%%%%%%%%%%%%%%%%%%%%%%%%%%%%%%%%%%
\begin{frame}{}
\usefont{T1}{ptm}{m}{n}\fontsize{60pt}{60pt}\selectfont
\begin{center}
        The Effective Potential
\end{center}
\end{frame}

%%%%%%%%%%%%%%%%%%%%%%%%%%%%%%%%%%%%%%%%%%%%%%%%%%%%%%%%%%%%%%%%%%
\begin{frame}{The Effective Scalar Potential}

	In {\sc M.Shifman, A.Yung arXiv:0803.0698} the effective scalar potential
	was found at large $ N $,

\begin{align*}
%
	V_\text{eff}  &  ~~=~~    -\, \frac{N}{4\pi} 
	\bigg\{
		\big( \big|\sqrt{2}\sigma\big|^2 \,+\, iD \big) 
		\log \big( \big|\sqrt{2}\sigma\big|^2 \,+\, iD \big)
		~-~
	\\
%
	&
	\qquad\qquad~
		~-~
		iD
		~-~
		\big|\sqrt{2}\sigma\big|^2 \log \big|\sqrt{2}\sigma\big|^2
	\bigg\}
\end{align*}

	Elimination of $ D $ leads to 
\[
	V_\text{eff}   ~~=~~ \frac{N}{4\pi}
	\bigg\{\,
		1 ~+~  \big|\sqrt{2}\sigma\big|^2\, \Big(\, \log \big|\sqrt{2}\sigma\big|^2 \,-\, 1 \,\Big)
	\,\bigg\}
\]
	This clearly does not fit into the $ \Sigma \, \log \Sigma $ picture!

\end{frame}




%%%%%%%%%%%%%%%%%%%%%%%%%%%%%%%%%%%%%%%%%%%%%%%%%%%%%%%%%%%%%%%%%%
%%%%%%%%%%%%%%%%%%%%%%%%%%%%%%%%%%%%%%%%%%%%%%%%%%%%%%%%%%%%%%%%%%
\section{Supersymmetric Form}
%%%%%%%%%%%%%%%%%%%%%%%%%%%%%%%%%%%%%%%%%%%%%%%%%%%%%%%%%%%%%%%%%%
%%%%%%%%%%%%%%%%%%%%%%%%%%%%%%%%%%%%%%%%%%%%%%%%%%%%%%%%%%%%%%%%%%
\begin{frame}{}
\usefont{T1}{ptm}{m}{n}\fontsize{50pt}{50pt}\selectfont
\begin{center}
        Supersymmetric Form
\end{center}
\end{frame}


%%%%%%%%%%%%%%%%%%%%%%%%%%%%%%%%%%%%%%%%%%%%%%%%%%%%%%%%%%%%%%%%%%
\begin{frame}{Supersymmetrizing}
	What is the supersymmetric form of these expressions?

	First, scatter them into series in
\[
	x    ~~=~~    \frac{iD}{\big|\sqrt{2}\sigma\big|^2}
\]
	as $ iD $ is naturally the supersymmetry breaking parameter,
\[
	 \frac{4\pi}{N}\, V_\text{eff}    ~~=~~
		x\, \big|\sqrt{2}\sigma\big|^2 \lgr -\, \ln\, \big|\sqrt{2}\sigma\big|^2 ~+~
		\sum_{k \geq 1}\, \frac{(-1)^k} 
                                      { k\,(k + 1) }\, x^k \rgr
\]
	Another representation is
\[
	- iD\, \ln \big|\sqrt{2}\sigma\big|^2 ~~+~~ \big|\sqrt{2}\sigma\big|^2\, \int_0^x \ln\, (1 + x)\, dx
\]
\end{frame}


%%%%%%%%%%%%%%%%%%%%%%%%%%%%%%%%%%%%%%%%%%%%%%%%%%%%%%%%%%%%%%%%%%
\begin{frame}{}

	Now we can promote $ x $ to superfields,
\[
	x    ~~=~~    \frac{iD}{\big|\sqrt{2}\sigma\big|^2}    ~~~~\longrightarrow~~~~
		\frac{S}{\sqrt{2}\Sigma}
\]
	where 
\[
	S    ~~=~~    \frac{i}{2}\,\ov D{}_R D_L \ln \sqrt{2}\ov\Sigma
\]
	and the lowest part of $ S \,/\, \Sigma $ is
\[
	\frac{S}{\sqrt{2}\Sigma} \bigg|    ~~=~~
		\frac{1}{\big|\sqrt{2}\sigma\big|^2}
		\lgr
			iD \,-\, F_{03}
			~~-~~
			\frac{2\, i\sqrt{2}\sigma \ov\lambda{}_R\lambda_L}{\big|\sqrt{2}\sigma\big|^2}
		\rgr
\]
\end{frame}


%%%%%%%%%%%%%%%%%%%%%%%%%%%%%%%%%%%%%%%%%%%%%%%%%%%%%%%%%%%%%%%%%%
\begin{frame}{}

	So we re-design the series in terms of superfields now
	(not straightforward)
\[
	\frac{i}{2}\, 
	\int\, d^2\tilde\theta\,
	S\,
	\sum_{k \geq 1}\, \frac{    (-1)^k    }
			   {  k\,(k + 1)\,(k + 2)  } \lgr \frac{S}{\sqrt{2}\Sigma} \rgr^k
\]

	This reproduces the original series 
\[
		x\, \big|\sqrt{2}\sigma\big|^2\,
		\sum_{k \geq 1}\, \frac{(-1)^k} 
                                      { k\,(k + 1) }\, x^k 
\]
	but the first term
\[
	-\, \frac{1}{2}\, \big|\sqrt{2}\sigma\big|^2\, x^2   ~~=~~  \frac{1}{2}\,\frac{D^2}{\big|\sqrt{2}\sigma\big|^2}
\]

\end{frame}

%%%%%%%%%%%%%%%%%%%%%%%%%%%%%%%%%%%%%%%%%%%%%%%%%%%%%%%%%%%%%%%%%%
\begin{frame}{Superkinetic term}
	This $ D^2 $ term comes from what we call a {\it superkinetic term}

\[
	\frac{1}{2}\,\frac{D^2}{\big|\sqrt{2}\sigma\big|^2}  ~~~\in~~~
	-\, \int\, d^4\theta\, \frac{1}{2}\, \Big| \ln \sqrt{2}\Sigma \, \Big|^2
\]

	Aside of $ D^2 $ it also contains the kinetic terms for $ F_{ik} $, $ \lambda $ and $ \sigma $

	The existence of this term has been known as far back as in\\
	{\sc A.~D'Adda, A.~.C.~Davis, P.~Di~Vecchia and P.~Salomonson,\\
	     Nucl. Phys. B 222, 45 (1983)}
\end{frame}


%%%%%%%%%%%%%%%%%%%%%%%%%%%%%%%%%%%%%%%%%%%%%%%%%%%%%%%%%%%%%%%%%%
\begin{frame}{}
	Putting together this superkinetic term, the Witten's potential
	and converting our series into a logarithm in a $ D $-term form,

\begin{align*}
%
	&
	\frac{4\pi}{N}\, \cell    ~~=~~     
			-\,
			\int\, d^4\theta\, \frac{1}{2}\, \Big| \ln \sqrt{2}\Sigma \, \Big|^2
			~~-~~
			i \int\, d^2\tilde\theta 
			\lgr
			\sqrt{2}\Sigma\, \ln \sqrt{2}\Sigma  ~-~ \sqrt{2}\Sigma
			\rgr
			~~-
	\\
%
			&
			-~~ 
			\frac{1}{4} \int\, d^4\theta\,
			\ln\, \sqrt{2}\ov\Sigma
			\lgr \Big( 1 \,+\, \frac{\sqrt{2}\Sigma}{S} \Big)^2\,
				\ln \Big(\, 1 \,+\, \frac{S}{\sqrt{2}\Sigma} \,\Big) ~-~
				\frac{\sqrt{2}\Sigma}{S} \rgr\!\!
			~+~ \text{h.c.}
\end{align*}

	Its component expansion is quite huge

	But the {\it constant} bosonic part of it does exactly become
\begin{align*}
%
	&
		-\, \big( \big|\sqrt{2}\sigma\big|^2 \,+\, iD \big) 
		\log \big( \big|\sqrt{2}\sigma\big|^2 \,+\, iD \big)
		~+~
		iD
		~+~
		\big|\sqrt{2}\sigma\big|^2 \log \big|\sqrt{2}\sigma\big|^2
		~~=
	\\
%
	& =~~
	- iD\, \ln \big|\sqrt{2}\sigma\big|^2 ~~+~~ \big|\sqrt{2}\sigma\big|^2\, \int_0^x \ln\, (1 + x)\, dx
\end{align*}

\end{frame}




%%%%%%%%%%%%%%%%%%%%%%%%%%%%%%%%%%%%%%%%%%%%%%%%%%%%%%%%%%%%%%%%%%
%%%%%%%%%%%%%%%%%%%%%%%%%%%%%%%%%%%%%%%%%%%%%%%%%%%%%%%%%%%%%%%%%%
\section{Component form}
%%%%%%%%%%%%%%%%%%%%%%%%%%%%%%%%%%%%%%%%%%%%%%%%%%%%%%%%%%%%%%%%%%
%%%%%%%%%%%%%%%%%%%%%%%%%%%%%%%%%%%%%%%%%%%%%%%%%%%%%%%%%%%%%%%%%%
\begin{frame}{}
\usefont{T1}{ptm}{m}{n}\fontsize{60pt}{60pt}\selectfont
\begin{center}
        Component form
\end{center}
\end{frame}


%%%%%%%%%%%%%%%%%%%%%%%%%%%%%%%%%%%%%%%%%%%%%%%%%%%%%%%%%%%%%%%%%%
\begin{frame}{Component expansion}

	We can think of the effective action by limiting to only {\it two}
	space-time derivatives -- this will yield a much more tractable expression\\[4mm]

	So we keep {\it all} powers of the auxiliary field $ D $, while
	retaining only {\it two} powers of the derivatives of {\it physical} fields
	$ \sigma $, $ \lambda $, $ A_k $\\[4mm]

	In that, we think of fermions as already having half a derivative
\end{frame}


\begin{frame}{}
\vskip -4mm
{\footnotesize
\begin{align*}
%
	\frac{4\pi}{N}\, \cell_\text{two deriv} &    ~~=~~  
	\frac{ \big|\p_\mu \sigma\big|^2 }
	{ \big|\sqrt{2}\sigma\big|^2 }
	~~-~~
	F_{03}\, \log\, \frac{\sqrt{2}\sigma}{\sqrt{2}\ov\sigma}
	~~+~~
	\frac{4\pi}{N}\, V_\text{eff}
	~~-
	\\[2mm]
%
	&
	~~-~~
	\frac{F_{03}^2}{\big|\sqrt{2}\sigma\big|^2}
	\lgr 2\, \frac{ \ln (1 + x) \,-\, x } { x^2 }  ~+~
		\frac{1}{2}\, \frac{1}{1 \,+\, x} \rgr
	~~-
	\\[2mm]
%
	&
	~~-~~
	\frac{
		\ov\lambda{}_R\, i\overleftrightarrow{\md_L} \lambda_R  ~+~ 
		\ov\lambda{}_L\, i\overleftrightarrow{\md_R} \lambda_L
	} { \big|\sqrt{2}\sigma\big|^2 }
	\,
	\frac{ \ln (1 + x) \,-\, x } { x^2 }
	~~-
	\\[2mm]
%
	&
	~~-~~ 
	2\,\frac{
		i\sqrt{2}\sigma\ov\lambda{}_R\lambda_L  ~+~  
		i\sqrt{2}\,\ov{\sigma\lambda}{}_L\lambda_R
	} { \big|\sqrt{2}\sigma\big|^2 }\,
	\frac{ \ln (1 + x) } { x }
	~~+
	\\[2mm]
%
	&
	~~+~~
	4\,\frac{
		\ov\lambda{}_R \lambda_L \ov\lambda{}_L \lambda_R
	} { \big|\sqrt{2}\sigma\big|^4 }\,
	\lgr \frac{ \ln (1 + x) \,-\, x } { x^2 }  ~+~
		\frac{1}{1 \,+\, x} \rgr
	~~+
	\\[2mm]
%
	&
	~~+~~
	\frac{1}{4}\,\Box\,\log |\sqrt{2}\sigma|^2 \cdot
	\frac{ (1 - x^2) \ln (1 + x) \,-\, x } { x^2 }
	~~-
	\\[3mm]
%
	&
	~~-~~
	2\,\frac{ F_{03} \big( i\sqrt{2}\sigma\ov\lambda{}_R\lambda_L ~-~
			       i\sqrt{2}\,\ov{\sigma\lambda}{}_L\lambda_R \big) }
		{ \big|\sqrt{2}\sigma\big|^4 }\,
	\frac{ \ln (1 + x) \,-\, x } { x^2 }\,.
\end{align*}
}
\end{frame}


%%%%%%%%%%%%%%%%%%%%%%%%%%%%%%%%%%%%%%%%%%%%%%%%%%%%%%%%%%%%%%%%%%
\begin{frame}{Truncation}

	This action is not supersymmetric --- field $ D $ which is sitting in $ x $
	effectively contains one space-time derivative, or two superspace derivatives
\[
	iD  ~~\in~~  \sqrt{2}\Sigma ~~=~~  \frac{i}{\sqrt 2}\, D_L\, \ov D{}_R\, V
\]

	So we to ``retain'' supersymmetry at the level of two space-time derivatives
	we have to put $ D ~=~ 0 $ wherever there are already two derivatives

\end{frame}


%%%%%%%%%%%%%%%%%%%%%%%%%%%%%%%%%%%%%%%%%%%%%%%%%%%%%%%%%%%%%%%%%%
\begin{frame}{}

	The result of this is 
\begin{align*}
%
\notag
	&     
	~~-~~ iD\, \log |\sqrt{2}\sigma|^2  ~~-~~  \frac{1}{2}\, \frac{ \big(i D\big)^2 } { \big|\sqrt{2}\sigma\big|^2 }
	~~-~~ F_{03}\, \log\, \frac{\sqrt{2}\sigma}{\sqrt{2}\ov\sigma}
	~~+~~
	\\[2mm]
%
\notag
	&
	~~+~~ \frac{
		\big|\p_\mu \sigma\big|^2  ~+~  \frac{\displaystyle 1}{\displaystyle 2}\,F_{03}^2  ~+~
		\frac{\displaystyle 1}{\displaystyle 2}\!
			    \lgr\, \ov\lambda{}_R\, i\overleftrightarrow{\md_L} \lambda_R  ~+~ 
				   \ov\lambda{}_L\, i\overleftrightarrow{\md_R} \lambda_L \rgr
		} { \big|\sqrt{2}\sigma\big|^2 }
	~~-~~
	\\[2mm]
%
\notag
	&
	~~-~~ 2\,\frac{
			i\sqrt{2}\sigma\ov\lambda{}_R\lambda_L  ~+~  
			i\sqrt{2}\,\ov{\sigma\lambda}{}_L\lambda_R
		} { \big|\sqrt{2}\sigma\big|^2 }
	~~+~~ 2\, \frac{
			\ov\lambda{}_R \lambda_L \ov\lambda{}_L \lambda_R
		} { \big|\sqrt{2}\sigma\big|^4 }
	~~+~~
	\\[2mm]
%
\notag
	&
	~~+~~
	\frac{	(iD + F_{03})\, i\sqrt{2}\sigma \ov\lambda{}_R\lambda_L ~+~
		(iD - F_{03})\, i\sqrt{2}\,\ov{\sigma \lambda}{}_L\lambda_R  }
		{ \big|\sqrt{2}\sigma\big|^4 }\,.
\end{align*}

	This matches the effective one-loop action calculated in {\sc M.Shifman, A.Yung arXiv:0803.0698}
\end{frame}


%%%%%%%%%%%%%%%%%%%%%%%%%%%%%%%%%%%%%%%%%%%%%%%%%%%%%%%%%%%%%%%%%%
\begin{frame}{Integrating $ D $}
	Another question we can ask is what does the action look like if 
	we leave only the physical fields --- $ A_k $, $ \sigma $ and $ \lambda $?\\[4mm]

	To do that we must return to the expression with $ \log(1 + x) \,-\, x $,
	and eliminate $ x $\\[4mm]

	This is not possible to do exactly --- but again, we only need two 
	space-time derivatives
\end{frame}


%%%%%%%%%%%%%%%%%%%%%%%%%%%%%%%%%%%%%%%%%%%%%%%%%%%%%%%%%%%%%%%%%%
\begin{frame}{}
	\vskip -6mm
{\footnotesize
\begin{align*}
%
	& \frac{4\pi}{N}\, \cell_\text{two deriv}(\sigma, A_\mu, \lambda)    ~~=~~  
	\\[2mm]
%
	& ~~=~~
	\frac{ \big|\p_\mu \sigma\big|^2 } r
	~~-~~
	F_{03}\, \log\, \frac{\sqrt{2}\sigma}{\sqrt{2}\ov\sigma}
	~~+~~
	v_\text{eff}(r)
	~~-
	\\[2mm]
%
	&
	~~+~~
	2\, F_{03}^2
	\lgr \frac{v_\text{eff}(r)}{(1 - r)^2}  ~-~  \frac{r}{4} \rgr
	~~+~~
	\lgr 
		\ov\lambda{}_R\, i\overleftrightarrow{\md_L} \lambda_R  ~+~ 
		\ov\lambda{}_L\, i\overleftrightarrow{\md_R} \lambda_L
	\rgr
	\frac{v_\text{eff}(r)}{(1 - r)^2}
	~~+
	\\[2mm]
%
	&
	~~+~~ 
	2 \lgr
		i\sqrt{2}\sigma\ov\lambda{}_R\lambda_L  ~+~  
		i\sqrt{2}\,\ov{\sigma\lambda}{}_L\lambda_R
	\rgr
	\frac{1 - r}{r}\, \ln r
	~~-
	\\[2mm]
%
	&
	~~-~~
	4\, \ov\lambda{}_R \lambda_L \ov\lambda{}_L \lambda_R
	\lgr
		\frac{v_\text{eff}(r)}{r\, (1 - r)^2}
		~-~  \frac{1}{r}
		~+~  \frac{r\, \big( 1 \,-\, r \,-\, \ln r \big)^2}{ (1 - r)^4 }
	\rgr
	~-
	\\[2mm]
%
	&
	~~-~~
	\frac{1}{4}\,\Box\,\ln r
	\lgr
		\frac{r\, v_\text{eff}(r)}{(1 - r)^2}  ~-~  \ln r
	\rgr
	~~+
	\\[2mm]
%
	&
	~~+~~
	2\, F_{03} \lgr 
			i\sqrt{2}\sigma\ov\lambda{}_R\lambda_L ~-~
			i\sqrt{2}\,\ov{\sigma\lambda}{}_L\lambda_R
		   \rgr
	\frac{v_\text{eff}(r)}{r\, (1 - r)^2}\,.
\end{align*}
	where $ r ~=~ \big|\sqrt{2}\sigma\big|^2 $ and
$
	v_\text{eff}(r)    ~~=~~    
	\frac{\displaystyle 4\pi}{\displaystyle N}\, V_\text{eff}(\sigma)    ~~=~~    
	r\, \ln r  ~~+~~  1  ~~-~~  r
$
}
\end{frame}




%%%%%%%%%%%%%%%%%%%%%%%%%%%%%%%%%%%%%%%%%%%%%%%%%%%%%%%%%%%%%%%%%%
%%%%%%%%%%%%%%%%%%%%%%%%%%%%%%%%%%%%%%%%%%%%%%%%%%%%%%%%%%%%%%%%%%
\section{Heterotic and Twisted mass deformations}
%%%%%%%%%%%%%%%%%%%%%%%%%%%%%%%%%%%%%%%%%%%%%%%%%%%%%%%%%%%%%%%%%%
%%%%%%%%%%%%%%%%%%%%%%%%%%%%%%%%%%%%%%%%%%%%%%%%%%%%%%%%%%%%%%%%%%
\begin{frame}{}
\usefont{T1}{ptm}{m}{n}\fontsize{60pt}{60pt}\selectfont
\begin{center}
        Heterotic and Massive deformations
\end{center}
\end{frame}


%%%%%%%%%%%%%%%%%%%%%%%%%%%%%%%%%%%%%%%%%%%%%%%%%%%%%%%%%%%%%%%%%%
\begin{frame}{Heterotic deformation and Twisted masses}

	Knowing the superfield form of the action it is straightforward
	to include the twisted masses
{\footnotesize
\begin{align*}
%
	& 4\pi\, \cell    ~~=~~     
			-\, \sum_k\, \Bigg\lgroup\,
			\int\, d^4\theta\, \frac{1}{2}\, \Big| \ln \big(\sqrt{2}\Sigma \,-\, m_k \big)\, \Big|^2
			~~+
	\\[2mm]
%
			&
			+~~
			i\, \int\, d^2\tilde\theta 
			\lgr
			\big( \sqrt{2}\Sigma \,-\, m_k \big)\, \ln \big( \sqrt{2}\Sigma \,-\, m_k \big)  ~-~ 
					\big( \sqrt{2}\Sigma \,-\, m_k \big)
			\rgr \!\!
			~~+
	\\[2mm]
%
			&
			+~~ 
			\frac{1}{4}\, \int\, d^4\theta\, \ln\, \big( \sqrt{2}\ov\Sigma \,-\, \ov m{}_k \big)\, \times
	\\[1mm]
%
			&
			\qquad\qquad~
			\times
			\bigg\lgroup \Big( 1 \,+\, \frac{\sqrt{2}\Sigma \,-\, m_k}{S_k} \Big)^2\,
				\ln \left(\, 1 \,+\, \frac{S_k}{\sqrt{2}\Sigma \,-\, m_k} \,\right) ~-~
				\frac{\sqrt{2}\Sigma \,-\, m_k}{S_k} \bigg\rgroup \Bigg\rgroup \!\!
			~~+
	\\[2mm]
%
			&
			+~~
			4\pi\, \int\, d^2\theta_R\,\, \hzeta_R\, \hbzeta_R
			~~-~~
			4\pi i\, \int d\theta_R\,\, \hzeta\cdot J(\sqrt{2}\hsigma)
			~~+~~
			\text{h.c.}
\end{align*}
}
	where
{\footnotesize
$
	\hsigma(\upsilon)    ~~=~~    \sigma  ~~-~~  \sqrt{2}\,\theta_R\,\ov\lambda{}_L
$
}
	and
{\footnotesize
$
	\hz(\upsilon)    ~~=~~    z  ~~-~~  \sqrt{2}\,\theta_R\,\zeta_L\
$
}

\end{frame}


%%%%%%%%%%%%%%%%%%%%%%%%%%%%%%%%%%%%%%%%%%%%%%%%%%%%%%%%%%%%%%%%%%
\begin{frame}{}

\usefont{T1}{ptm}{m}{n}\fontsize{70pt}{70pt}\selectfont
\begin{center}
        Thank you
\end{center}

\end{frame}



\end{document}
